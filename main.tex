\documentclass[11pt, a4paper, twocolumn]{article} %\documentclass[11pt, a4paper, twoside, openright]{article} %draft

\usepackage{graphicx,color}
\usepackage{amssymb, amsmath, array}
\usepackage{hyperref}
\usepackage{newpxtext, newpxmath}


\begin{document}

\onecolumn

\input{cover}


\thispagestyle{empty}

\newpage

\tableofcontents{\protect\thispagestyle{empty}

\twocolumn

\clearpage

\section{Introduction}

Explain the importance of public-key cryptography, and why big numbers
are used.

Explain why timing leaks are a problem, and that big numbers are hard
to implement in constant-time.

Explain what work we've done to address this, and what performance
results we've gotten.

\section{Background}

Explain the necessary math and concerns that are relevant for our project.

\subsection{Big Numbers in Cryptography}

Explain how Big Numbers are used in Public Key Cryptography. Mention
DSA, RSA, Elliptic curves.

\subsubsection{Big Numbers in \texttt{go/crypto}}

Explain how things are used in Go's standard library.

\subsection{Side-Channels}

Explain the concept of side-channels in cryptography, and give some of
the fundamental types we're concerned with.

\subsubsection{Actual Attacks}

Explain how these concerns actually lead to vulnerabilities in systems.

\subsubsection{Our Threat-Model}

Explain the threat model we have, and what side-channels we aren't concerned
about.

\subsection{Vulnerabilities in \texttt{big.Int}}

Explain what vulnerabilities are potentially lurking in bigInt.

\subsubsection{Padding and Truncation}

Explain how big.Int truncates numbers internally.

Explain potential issues with padding in cryptography.

\subsubsection{Leaky Algorithms}

Explain how most algorithms are potentially leaky.

\subsubsection{Mitigations}

Explain what mitigations are deployed in Go.

\section{Implementation}

Describe at a high level what we've done.

\subsection{Strategies for numbers in Cryptography}

Describe different strategies in place for providing numbers for
Cryptography.

\subsection{The \texttt{safenum} library}

Describe at a high level what the library provides.

\subsubsection{Handling Size}

Describe how we handle sizing of numbers.

\subsection{Some Basic Techniques}

Describe some basic techniques for constant-time operation.

\subsection{Some Algorithm Choices}

Describe the algorithm choices we've used for different things.

\section{Results}

Describe what results we've managed to perform.

\subsection{Comparison with \texttt{big.Int}}

Describe the final performance results we've managed to achieve.

\subsection{Comparison with \texttt{go/crypto}}

Describe the benchmarks on actual code.

\section{Further Work}

\subsection{Upstreaming to \texttt{go/crypto}}

Describe our work in providing a patch for RSA, and what results we've
managed to achieve.

\section{Conclusion}

Summarize the things we put in the introduction.

\section*{Acknowledgements}
\addcontentsline{toc}{section}{Acknowledgements}

\bibliographystyle{alpha}
\bibliography{references}
\end{document}
