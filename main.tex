\documentclass[11pt, a4paper, twocolumn]{article} %\documentclass[11pt, a4paper, twoside, openright]{article} %draft

\usepackage{graphicx,color}
\usepackage{amssymb, amsmath, array}
\usepackage{hyperref}
\usepackage{newpxtext, newpxmath}


\begin{document}

\onecolumn

\input{cover}


\thispagestyle{empty}

\newpage

\tableofcontents{\protect\thispagestyle{empty}

\twocolumn

\clearpage

\section{Introduction}

With the Internet cemented as a principal keystone in communication, and ever
increasing activity taking place digitially, the importance of secure
communication and Cryptography has never been greater. Thankfully, after
50 years of Public Key Cryptography
\cite{hellman_overview_1978},
we have good theoretical systems to provide these guarantees.

Most of these systems rely on modular arithmetic with large numbers,
such as RSA or Elliptic Curve Cryptography (CITATION).
Working with such numbers is not natively supported by hardware,
requiring a "Big Number" software library to provide this functionality.

Unfortunately, even though Public Key Cryptosystems have been heavily
scrutinized \textit{in theory}, in practice many vulnerabilities arise
in software implementations of these systems.

One particularly pernicious class of vulnerability are
\textbf{timing attacks}
\cite{kocher_cryptanalysis_1995}, where an implementation leaks information
about secret values through its execution time or cache usage, among
many side-channels.

Libraries for Big Numbers that are not designed with Cryptography in mind
are pervasively vulnerable to this class of attack.

In particular, Go \cite{go_language} provides a general purpose
Big Number type, \texttt{big.Int}, which suffers from these vulnerabilities,
as we detail later in this report. Unfortunately, this library
gets used for Cryptography
\cite{ford_proposal_2017}, including inside of Go's own standard library,
in \texttt{go/crypto}.

We've addressed this issue by creating a library
\cite{safenum}
designed to work with Big Numbers in the context of Public Key Cryptography.
Our library provides the necessary operations for implementing these systems,
all while avoiding the leakage of secret information.
To demonstrate its utility, we've modified Go's \texttt{go/crypto}
package, replacing the use of \texttt{big.Int} in the DSA and RSA
systems.

\section{Background}

Explain the necessary math and concerns that are relevant for our project.

\subsection{Big Numbers in Cryptography}

Explain how Big Numbers are used in Public Key Cryptography. Mention
DSA, RSA, Elliptic curves.

\subsubsection{Big Numbers in \texttt{go/crypto}}

Explain how things are used in Go's standard library.

\subsection{Side-Channels}

Explain the concept of side-channels in cryptography, and give some of
the fundamental types we're concerned with.

\subsubsection{Actual Attacks}

Explain how these concerns actually lead to vulnerabilities in systems.

\subsubsection{Our Threat-Model}

Explain the threat model we have, and what side-channels we aren't concerned
about.

\subsection{Vulnerabilities in \texttt{big.Int}}

Explain what vulnerabilities are potentially lurking in bigInt.

\subsubsection{Padding and Truncation}

Explain how big.Int truncates numbers internally.

Explain potential issues with padding in cryptography.

\subsubsection{Leaky Algorithms}

Explain how most algorithms are potentially leaky.

\subsubsection{Mitigations}

Explain what mitigations are deployed in Go.

\section{Implementation}

Describe at a high level what we've done.

\subsection{Strategies for numbers in Cryptography}

Describe different strategies in place for providing numbers for
Cryptography.

\subsection{The \texttt{safenum} library}

Describe at a high level what the library provides.

\subsubsection{Handling Size}

Describe how we handle sizing of numbers.

\subsection{Some Basic Techniques}

Describe some basic techniques for constant-time operation.

\subsection{Some Algorithm Choices}

Describe the algorithm choices we've used for different things.

\section{Results}

Describe what results we've managed to perform.

\subsection{Comparison with \texttt{big.Int}}

Describe the final performance results we've managed to achieve.

\subsection{Comparison with \texttt{go/crypto}}

Describe the benchmarks on actual code.

\section{Further Work}

\subsection{Upstreaming to \texttt{go/crypto}}

Describe our work in providing a patch for RSA, and what results we've
managed to achieve.

\section{Conclusion}

Summarize the things we put in the introduction.

\section*{Acknowledgements}
\addcontentsline{toc}{section}{Acknowledgements}

\bibliographystyle{plainurl}
\bibliography{references}
\end{document}
